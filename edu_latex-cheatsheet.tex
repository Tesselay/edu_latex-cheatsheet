\documentclass[12pt,landscape,a4paper]{article}
\usepackage{multicol}
\usepackage[ngerman]{babel}
\usepackage[utf8]{inputenc}
\usepackage[T1]{fontenc}
\usepackage[nosf]{kpfonts}
\usepackage[t1]{sourcesanspro}
\usepackage[landscape, top=1mm, left=1mm, right=1mm, bottom=2mm]{geometry}
\usepackage{microtype}
\usepackage[framemethod=tikz]{mdframed}
\usepackage{spverbatim}
\usepackage{array}
\usepackage{tasks}
\usepackage{exsheets}
\usepackage{graphicx}
\usepackage{caption}
\usepackage{subcaption}
\usepackage{verbdef}

\usepackage{lipsum}

\pagestyle{empty}

\setcounter{secnumdepth}{0}
\setlength{\parindent}{0pt}
\setlength{\parskip}{0pt plus 0.5ex}
\setlength{\columnsep}{2mm}
\graphicspath{ {F:/Schule/InfV/CheatSheet/edu_latex-cheatsheet/images/} }


\begin{document}
\raggedright
\footnotesize
\begin{multicols}{4}
    \begin{center}
        \textbf{introduction}

        \textit{\TeX\: \& METAFONT were developed by Dr. Donald E. Knuth and were Open Source since 1980. It is a typesetting 
        application but an own programming language with over 300 commands as well. Everything is handled as a block. 
        It supports multiple diacritics, dialects, etc. and was finished 1990 with version 3.14. Bugs haven't been found for a 
        long time. It's OS \& device independent and uses the .tex format, which is a DVI (Device-Independent) format. It has 
        to be programmed for own use but there exist countless packages.}

        \vspace*{4mm}

        \textit{One of those packages is \LaTeX, originally developed by Leslie Lamport. It's a macropackage, tying up \TeX commands to 
        custom markup commands. Additionally, it gives a default for formatting and layout. It has become the standard for scientific 
        publications, also due to its functionality and remarkable line- and pagebreak algorithm. It's current version is $2^e$, 3 has
        been in development since 1989}

    \end{center}
    \begin{center}
        \textbf{essentials}

        \spverb|\documentclass[options]{styles}| 
        
        \vspace*{1mm}
        \begin{tabular*}{5cm}{cl}
            fleqn & left align math \\
            leqno & left align math num \\
            \hline 
            article & no \spverb|\part| or \spverb|\chapter| \\
            IEEEtran & (one) IEEE standard \\
            minimal & min formatting \\
            slides & presentation 
        \end{tabular*}

        \vspace*{2mm}

        \spverb|\usepackage[options]{package}|
    \end{center}
    \begin{center}
        \textbf{structure}

        \textit{Between documentclass and document is} \textbf{Preamble}\textit{, documentwide commands come there.}

        \vspace*{2mm}

        \textbf{structure macros}\textit{ end at begin of new structure/end of document}

        \vspace*{1mm}
        
        \begin{tabular*}{7cm}{cl}
            \textbf{level} & \textbf{macro} \\
            -1 & \spverb|\part{name}| \\
            0 & \spverb|\chapter{name}| \\
            1 & \spverb|\section{name}| \\
            2 & \spverb|\subsection{name}| \\
            3 & \spverb|\subsubsection{name}| \\
            4 & \spverb|\paragraph{name}| \\
            5 & \spverb|\subparagraph{name}| 
        \end{tabular*} \newline \newline

        \textbf{structure environments}\textit{ use common block syntax}

        \begin{tabular*}{7cm}{b{4cm}b{3cm}}
            \begin{spverbatim}\begin{document} \end{document}\end{spverbatim} 
            & All printed content \\
            \begin{spverbatim}\begin{abstract} \end{abstract}\end{spverbatim} 
            & Short synopsis (has special layout) \\
            \begin{spverbatim}\begin{quote} \end{quote}\end{spverbatim} 
            & Special layout for quotes \\
            \begin{spverbatim}{ }\end{spverbatim} 
            & custom block
        \end{tabular*}

        \vspace*{2mm}

        \textbf{Page style} \textit{defines footer \& header display}

        \begin{tabular*}{7cm}{cp{3cm}}
            \spverb|\thispagestyle{empty}| & no header \& footer \\
            \spverb|\thispagestyle{plain}| & same but page number shown \\
            \spverb|\thispagestyle{headings}| & same but maybe header (depending on class) \\
            \spverb|\thispagestyle{myheadings}| & setup of own h \& f 
        \end{tabular*}

        \vspace*{2mm}

        \textit{Additional}\textbf{ structure manipulation commands}\textit{ exist. First 4 go into Preamble.}

        \begin{tabular*}{7cm}{m{3.5cm}m{3.5cm}}
            \begin{spverbatim}\renewcommand {\abstractname}{name}\end{spverbatim} & name of abstract \\
            \vspace*{-3mm}
            \begin{spverbatim}\setcounter {secnumdepth}{n}\end{spverbatim} & depth for section numbering \\
            \vspace*{-3mm}
            \begin{spverbatim}\setcounter{tocdepth} {n}\end{spverbatim} & depth of sections in table \\
            \vspace*{-3mm}
            \spverb|\setcounter{page}{n}| & reset counter to n \\
            \vspace*{-2mm}
            \spverb|\pagemark| & prints pagenumber \\
            \begin{spverbatim}\addcontentsline{toc} {subsection}{name}\end{spverbatim} & force add structure to table. Place in doc defines place in table \\
        \end{tabular*}

    \end{center}
    \begin{center}
        \textbf{titlepage}

        \textit{Content to the title can be provided through:}
        \spverb|\title{string}|
        \spverb|\date{date}|
        \spverb|\author{author \and author2 \thanks{text}}|

        \vspace*{2mm}

        \spverb|\maketitle|

        \vspace*{2mm}

        \textit{A custom titlepage can be generated as well, though everything needs to be done by hand then}
        \begin{spverbatim}
            \begin{titlepage}
                ...
            \end{titlepage}
        \end{spverbatim}
    \end{center}
    \begin{center}
        \vspace*{-4mm}
        \textbf{contenttables}

        \begin{tabular*}{6cm}{cl}
            \spverb|\tableofcontents| & section types \\
            \spverb|\listoftables| & table envs \\
            \spverb|\listoffigures| & figure envs
        \end{tabular*}

        \vspace*{2mm}

        \textit{content can be omitted from content tables by adding * to macro name}

        e.g. \spverb|\section*{}|

        \vspace*{2mm}

        \textit{different name for tables and inside document can be provided}

        \spverb|\section[table name]{document name}|
    \end{center}
    \begin{center}
        \subsubsection*{formatting}
    \end{center}
    \begin{center}
        \vspace*{-1mm}
        \textbf{text style}

        \begin{tabular*}{5cm}{cl}
            \spverb|\textbf{text}| & \textbf{bold} \\
            \spverb|\textit{text}| & \textit{cursive} \\
            \spverb|\texttt{text}| & \texttt{typewriter} \\
            \spverb|\textsc{text}| & \textsc{small capitals} \\
            \spverb|\textrm{text}| & \textrm{roman font} \\
            \spverb|\textsf{text}| & \textsf{serif font} \\
            \spverb|\textnormal{text}| & \textnormal{default of doc} \\
        \end{tabular*}
    \end{center}
    \begin{center}
        \textbf{text size}

        \textit{Need to be placed inside blocks/envs.}

        \begin{tabular*}{4.6cm}{c}
            \spverb|\Huge| \\
            \spverb|\huge| \\ 
            \spverb|\LARGE| \\
            \spverb|\Large| \\
            \spverb|\large| \\
            \spverb|\normalsize| \\
            \spverb|\small| \\
            \spverb|\footnotesize| \\
            \spverb|\tiny| \\
            \spverb|\scalebox{scale}{text}| 
        \end{tabular*}
    \end{center}
    \begin{center}
        \textbf{special characters}

        \textit{Most special characters can be escaped.}

        \begin{tabular*}{5cm}{cl}
            \spverb|\textbackslash| & \textbackslash \\
            \spverb|\^| & \^s \\
            \spverb|\dq| & \dq \\
            \spverb|\textbar| & \textbar \\
            \spverb|\textless| & \textless \\
            \spverb|\textgreater| & \textgreater \\
            \spverb|\LaTeX| & \LaTeX \\
            \spverb|\TeX| & \TeX \\
        \end{tabular*}
    \end{center}
    \begin{center}
        \textbf{newline}

        \begin{tabular*}{5cm}{cl}
            \spverb|\\| & linebreak, no par \\
            \spverb|\newline| & same \\
            \: & linebreak, new par \\
            \spverb|\par| & same \\
        \end{tabular*}
    \end{center}
    \begin{center}
        \textbf{justification \& space}

        \begin{tabular*}{6cm}{c p{3cm}}
            \spverb|\centering| & centers block \\
            \spverb|\raggedleft| & rights block \\
            \spverb|\raggedright| & lefts block \\
            \spverb|\begin{center}..| & new centered block \\       
            \hline
            \spverb|\,| & 0.16667em space \\
            \spverb|\:| & default space \\
            \spverb|~| & newline-safe def \\
            \spverb|\hspace{n}| & n horizontal space \\
            \hline
            \spverb|\smallskip| & small vert space \\
            \spverb|\medskip| & .. \\
            \spverb|\bigskip| & .. \\
            \spverb|\vspace{n}| & n vertical space \\
            \hline 
            \spverb|\newpage| & Ends column \\
            \spverb|\clearpage| & Ends page \\
            \spverb|\vfill| & Justifies content vertically around \\
            \spverb|\hfill| & Justifies content horizontally around 
        \end{tabular*}
    \end{center}
    \begin{center}
        \textbf{misc}

        \begin{tabular*}{7cm}{m{4cm}m{2.5cm}}
            \spverb|\color{color!opacity}| & color of block \\
            \hline
            \spverb|\noindent| & removes indent on line\\
            \vspace*{-3mm}\begin{spverbatim}\setlength{\parindent} {n}\end{spverbatim} & sets indent for whole document \\
            \hline 
            \spverb|\rule{width}{thickness}| & horizontal line \\
            \vspace*{-3mm}\begin{spverbatim}\noindent\makebox[ \linewidth]{\rule{ \paperwidth}{0.4pt}}\end{spverbatim} & document spanning line \\
            \hline 
            \textbackslash verb\textbar arg\textbar & inline macro display \\
            \vspace*{-3mm}\begin{spverbatim}\begin{verbatim} .. \end{verbatim}\end{spverbatim} & multiline macro display \\
        \end{tabular*}
    \end{center}
    \begin{center}
        \textbf{packages}

        \begin{tabular*}{7cm}{p{1.2cm}l}
            babel & \spverb|\usepackage[sub, main]{babel}| \\
            inputenc & Input enc (def=ascii, alt=utf8) \\
            fontenc & Font enc (def=ot1, alt=t1) \\
            hyperref & \spverb|\url{add}, \href{add}{dname}| \\
            geometry & custom formatting \\
            scrlayer-scrpage & \spverb|\ihead{} \chead{} \ohead{}|
        \end{tabular*}
    \end{center}
    \begin{center}
        \textbf{lists}
        

        \begin{tabular*}{7cm}{b{4cm}b{3cm}}
            \begin{spverbatim}\begin{enumerate} \item entry 1 \item entry 2 \end{enumerate}\end{spverbatim} 
            & \begin{enumerate} \item entry 1 \item entry 2 \end{enumerate} \\
            \hline
            \begin{spverbatim}\begin{itemize}    \item entry 1    \item entry 2 \end{itemize}\end{spverbatim}
            & \begin{itemize} \item entry 1 \item entry 2 \end{itemize} \\
            \hline
            \begin{spverbatim}\begin{description} \item [name1] entry 1 \item [name2] entry 2 \end{description}\end{spverbatim}
            & \begin{description} \item [name1] entry 1 \item [name2] entry 2 \end{description} \\
            \hline
        \end{tabular*}

        \textit{Can be used as horizontal list. Needs tasks and exsheets package.}
        \vspace*{-4mm}
        \begin{spverbatim}\begin{question} question \begin{tasks}(columns) \task task 1 \task task 2 \end{tasks} \end{question}\end{spverbatim}
        \vspace*{-4mm}
        \begin{question} task question? \begin{tasks}(2) \task answer 1 \task answer 2 \end{tasks} \end{question} 
        \vspace*{-7mm}
        \rule{7cm}{0.4pt}
        \textit{Lists of same type can be nested as well. They'll have different stylization.}
        \vspace*{-5mm}
        \begin{spverbatim}\begin{enumerate} \item entry 1 \begin{itemize} \item entry 1.1 \item entry 1.2 \end{itemize} \item entry 2 \end{enumerate}\end{spverbatim}
        \vspace*{-4mm}
        \begin{enumerate} \item entry 1 \begin{itemize} \item entry 1.1 \item entry 1.2 \end{itemize} \item entry 2 \end{enumerate}
        \vspace*{-4mm}
        \rule{7cm}{0.4pt}
    \end{center}
    \begin{center}
        \subsubsection*{floats}

        \textit{Floats are containers, that won't be broken over multiple pages.}

        \begin{spverbatim}
        \begin{float}[placement]
            ...
        \end{float}
        \end{spverbatim}

        \begin{tabular*}{6.5cm}{c|cl}
            \textbf{types} & \multicolumn{2}{c}{placement} \\
            \hline
            table & h & here \\
            figure & t & top \\
            subfigure & b & bottom \\
            & p & page \\
            & ! & override default \\
            & H & precise here (float pkg) \\
        \end{tabular*}

        \vspace*{3mm}

        \spverb|\caption[short]{title}|

        \spverb|\label{name}|

        \spverb|\ref{labelName}|

        \textit{For better label naming, a "type:" can be added before the name.}

        \begin{tabular*}{6.5cm}{cccccc}
            \multicolumn{6}{c}{\textbf{type examples}} \\
            ch: & sec: & subsec: & fig: & tab: & eq: \\
            lst: & itm: & alg: & app: 
        \end{tabular*}
    \end{center}
    \begin{center}
        \textbf{graphics}

        \spverb|\usepackage{graphicx}|

        \vspace*{2mm}

        \textit{Either provide a image folder and only import the name (on win: / instead of \textbackslash): }

        \spverb|\graphicspath{path}|

        \spverb|\includegraphics{name}|

        \vspace*{2mm}

        \textit{or provide the path directly:}

        \spverb|\includegraphics[options]{path}|

        \vspace*{2mm}

        \begin{tabular*}{6.5cm}{cccc}
            \multicolumn{4}{c}{\textbf{options}} \\
            scale & width & height & angle \\
            draft & keepaspectratio & &   
        \end{tabular*}

    \end{center}
    \begin{center}
        \textbf{figure}

        \textit{figure is an environment commonly used for images. It provides labels and captions.}

        \begin{spverbatim}
        \begin{figure}[placement]
            ... 
        \end{figure}
        \end{spverbatim}

        \textit{More complex figures can be created with the subfigure environment. Needs the caption 
        and subcaption package. Can be used to put multiple images in a row, etc.}

        \begin{spverbatim}
        \begin{figure} 
            \begin{subfigure}
                ...
            \end{subfigure}
            ...
        \end{figure}
        \end{spverbatim}
        
    \end{center}
    \begin{center}
        \textbf{tables}

        \textit{The \textbackslash table float is not necessary but provides label \& captions.}

    \begin{spverbatim}
    \begin{table}[placement]
    \begin{tabular}[position]{layout}
        content  
    \end{tabular}
    \end{table}    
    \end{spverbatim}
    
    \begin{tabular*}{7cm}{m{2cm}p{4.5cm}}
        \multicolumn{2}{c}{\textbf{Pos (mostly useless)}} \\
        b & bottom \\
        c & center \\
        t & top \\
        \hline 
        \multicolumn{2}{c}{\textbf{Layout}} \\
        l & left justified \\
        c & center justified \\
        r & right justified \\
        p\{n\} & n wide line breakable cell. Align with top of row \\
        m\{n\} & same. Align with center \\ 
        b\{n\} & same. Align with bottom \\
        | & vertical line \\
        || & double line \\
        \hline 
        \multicolumn{2}{c}{\textbf{content layout}} \\
        \& & col seperator \\ 
        \textbackslash\textbackslash & new row \\
        \textbackslash hline & horizontal line \\
        \textbackslash newline & new line within cell \\
        \textbackslash cline\{i-j\} & line begin col i \& end col j \\
        \vspace*{-4mm}
        \begin{spverbatim}\multicolumn {n} {layout} {content}\end{spverbatim} & content across n columns in row 
    \end{tabular*}

    \end{center}
    \begin{center}
        \subsubsection*{mathematics}

        \begin{tabular*}{6.5cm}{b{2cm}b{4cm}}
            \& & justify formulas to that pos \\
            \textbackslash nonumber & line not counted as eq \\
            \vspace*{-4mm}
            \begin{spverbatim}\begin{split} \end{split}\end{spverbatim} & use in singleine env for multiline 
        \end{tabular*}

        \vspace*{2mm}

        \textit{numbering can be omitted with an * added, like \textbackslash begin\{align*\}. Labels in
        multiline envs need to added to every wanted line for referencing.}\newline\newline\newline

    \end{center}
    \begin{center}
        \textbf{environments}

        \begin{tabular*}{7cm}{b{3.5cm}ccc}
            \textbf{env} & \textbf{d/t} & \textbf{sl/ml} & \textbf{num} \\
            \vspace*{-4mm}
            \begin{spverbatim}$..$ \end{spverbatim} & t & sl & n \\
            \vspace*{-4mm}
            \begin{spverbatim}$$..$$ \end{spverbatim} & d & sl & n \\
            \vspace*{-4mm}
            \begin{spverbatim}\(..\) \end{spverbatim} & t & sl & n \\
            \vspace*{-4mm}
            \begin{spverbatim}\[..\] \end{spverbatim} & d & sl & n \\
            \vspace*{-4mm}
            \begin{spverbatim}\begin{math}.. \end{math}\end{spverbatim} & t & sl & n \\
            \vspace*{-4mm}
            \begin{spverbatim}\begin{displaymath}.. \end{displaymath}\end{spverbatim} & d & sl & n \\
            \vspace*{-4mm}
            \begin{spverbatim}\begin{equation}.. \end{equation}\end{spverbatim} & d & sl & y \\
        \end{tabular*}

        \textit{amsmath envs are all d, ml and num.}

        \begin{tabular*}{7cm}{b{3.5cm}b{3.5cm}}
            \vspace*{-4mm}
            \begin{spverbatim}\begin{multiline}.. \end{multiline}\end{spverbatim} & only for single eqs too long for 1 line \\
            \vspace*{-4mm}
            \begin{spverbatim}\begin{gather}.. \end{gather}\end{spverbatim} & newlines like tabular, no align \\
            \vspace*{-4mm}
            \begin{spverbatim}\begin{align}.. \end{align}\end{spverbatim} & newlines like tabular, align \\
        \end{tabular*}
    \end{center}
    \begin{center}
        \textbf{functions}

        \begin{tabular*}{7cm}{cc|cc}
            \spverb|^{x}| & $1^{x}$ & \spverb|_{x}| & $1_{x}$ \\
            \hline
            \spverb|\sqrt[x]{y}| & $\sqrt[x]{y}$  & \spverb|(x+y)| & $(x+y)$ \\ 
            \hline
            \spverb|\frac{x}{y}| & $\frac{x}{y}$ & \spverb|\{x+y\}| & $\{x+y\}$ \\
            \hline
            \spverb|\binom{x}{y}| & $\binom{x}{y}$ & \spverb|[x+y]| & $[x+y]$ \\
            \hline
        \end{tabular*}
        \begin{tabular*}{7cm}{c|c}
            \spverb|overbrace{x}| & $\overbrace{x}$ \\
            \hline
            \spverb|underbrace{x}| & $\underbrace{x}$ \\
            \hline
        \end{tabular*}
        \begin{tabular*}{7cm}{b{4cm}b{3cm}}
            \vspace*{-4mm}
            \begin{spverbatim}\leftTYPE .. \rightTYPE\end{spverbatim} & $\left>40\right<$ $\left<40\right>$ $\left|40\right|$\\
            \hline
            \begin{spverbatim}\cfrac{x}{y}\end{spverbatim} & $\cfrac{x}{x + \cfrac{x}{y + \cfrac{x}{y+n}}}$\\
            \hline
            \multicolumn{2}{c}{\textit{Both need an env. Almost identical.}} \\
            \vspace*{-4mm}
            \begin{spverbatim}\begin{array} {layout}.. \end{array}\end{spverbatim} & \begin{spverbatim}\begin{matrix}.. \end{matrix}\end{spverbatim} 
        \end{tabular*}

        \vspace*{-6mm}
       
        \begin{gather}\begin{matrix} z & = & a \\ & = & a \\ f(x,y,z) & = & x + y + z \end{matrix}\end{gather}

        \vspace*{-6mm}

        \textit{with amsmath a t or d can be added to some functions for better textmode / displaymode display (e.g. \textbackslash dfrac).}


    \end{center}

\end{multicols}
\end{document}




